
\documentclass[12pt]{report}
\usepackage{textcomp}

\addtolength{\hoffset}{-2.25cm}
\addtolength{\textwidth}{4.5cm}
\addtolength{\voffset}{-2.5cm}
\addtolength{\textheight}{5cm}
\setlength{\parskip}{0pt}
\setlength{\parindent}{15pt}

\usepackage{graphicx}
\graphicspath{ {./images/} }

\usepackage{amsthm}
\usepackage{amsmath}
\usepackage{amssymb}
\usepackage{dutchcal}

\usepackage{environ}
\NewEnviron{eq}{
\begin{equation}\begin{split}
  \BODY
\end{split}\end{equation}
    }

\usepackage{blkarray}

\newcommand{\be}{\beta}
\newcommand{\ga}{\gamma}

\usepackage[colorlinks = true, linkcolor = black, citecolor = black, final]{hyperref}
\hypersetup{
    colorlinks,
    citecolor=black,
    filecolor=black,
    linkcolor=black,
    urlcolor=black
}

\usepackage{graphicx}
\usepackage{multicol}
\usepackage{marvosym}
\usepackage{wasysym}

\usepackage{outlines}
\usepackage[utf8]{inputenc}
\usepackage[english]{babel}

\usepackage{enumitem,amssymb}

\newcommand{\pr}[1]{\left(#1\right)}
\newcommand{\cross}{\times}
\newcommand{\grad}{\nabla}

\usepackage{pgfplots}
\pgfplotsset{compat=1.16}
\usepackage{circuitikz}
\usepackage{tikz}
\usetikzlibrary{patterns}
\usetikzlibrary{arrows.meta}

\usepackage{mdframed}
\usepackage{pbox}
\usepackage{array}
\usepackage{makecell}
\usepackage{booktabs}
\setlength{\heavyrulewidth}{1pt}
\setlength{\abovetopsep}{4pt}


\setlength{\parindent}{0in}
\usepackage{changepage}

\pagestyle{plain}

\definecolor{codegreen}{rgb}{0,0.6,0}
\definecolor{codegray}{rgb}{0.5,0.5,0.5}
\definecolor{codepurple}{rgb}{0.58,0,0.82}
\definecolor{backcolour}{rgb}{0.95,0.95,0.92}
\definecolor{backcolour2}{rgb}{0.44,0.63,0.99}

\definecolor{dkgreen}{rgb}{0,0.6,0}
\definecolor{gray}{rgb}{0.5,0.5,0.5}
\definecolor{mauve}{rgb}{0.58,0,0.82}

\usepackage{listings}
\lstset{
    language=Verilog,
    breakatwhitespace=false,         
    breaklines=true,                 
    captionpos=b,                    
    keepspaces=true,                 
    numbers=left,                    
    numbersep=5pt,                  
    showspaces=false,                
    showstringspaces=false,
    showtabs=false       
    columns=flexible,
    basicstyle={\small\ttfamily},
    numberstyle=\tiny\color{gray},
    keywordstyle=\color{blue},
    commentstyle=\color{dkgreen},
     % stringstyle=\color{mauve},
    stringstyle=\color{orange},
    backgroundcolor=\color{backcolour},
    tabsize=4  
}

\lstnewenvironment{CPP}
  {\lstset{language=C,basicstyle=\ttfamily\small,frame=none}}
  {}

\usepackage{lmodern}
\usepackage{afterpage}
\usepackage{adjustbox}
\usepackage{helvet}

\counterwithout{footnote}{chapter} % makes footnotes count over document rather than by chapter

%%%%%%%%%%%%%%%% These commands are taken from Professor Ashmanskas!
\newcommand{\ohm}{\Omega}
\newcommand{\kohm}{{\rm k}\Omega}
\newcommand{\Mohm}{{\rm M}\Omega}
\newcommand{\dif}{{\rm d}}
\newcommand{\kgmps}{{\rm kg\cdot m/s}}
\newcommand{\degC}{{}^\circ{\rm C}}
\newcommand{\degF}{{}^\circ{\rm F}}
\newcommand{\K}{{\rm K}}
\newcommand{\liter}{{\rm L}}
\newcommand{\mL}{{\rm mL}}
\newcommand{\atm}{{\rm atm}}
\newcommand{\mol}{{\rm mol}}
\newcommand{\Cal}{{\rm Cal}}
\newcommand{\W}{{\rm W}}
\newcommand{\kW}{{\rm kW}}
\newcommand{\dB}{{\rm dB}}
\newcommand{\amu}{{\rm u}}
\newcommand{\Pa}{{\rm Pa}}
\newcommand{\J}{{\rm J}}
\newcommand{\N}{{\rm N}}
\newcommand{\pC}{{\rm pC}}
\newcommand{\C}{{\rm C}}
\newcommand{\T}{{\rm T}}
\newcommand{\eV}{{\rm eV}}
\newcommand{\V}{{\rm V}}
\newcommand{\mV}{{\rm mV}}
\newcommand{\Vpp}{{ V}_{ pp}}
\newcommand{\mVpp}{{\rm mV}_{\rm pp}}
\newcommand{\Vac}{{\rm V}_{\rm AC}}
\newcommand{\Vdc}{{\rm V}_{\rm DC}}
\newcommand{\Vbold}{{\bf V}}
\newcommand{\Ibold}{{\bf I}}
\newcommand{\Zbold}{{\bf Z}}
\newcommand{\kV}{{\rm kV}}
\newcommand{\emf}{{\cal E}}
\newcommand{\A}{{\rm A}}
\newcommand{\mA}{{\rm mA}}
\newcommand{\uA}{{\rm \mu A}}
\newcommand{\nA}{{\rm nA}}
\newcommand{\F}{{\rm F}}
\newcommand{\uF}{\mu{\rm F}}
\newcommand{\nF}{{\rm nF}}
\newcommand{\pF}{{\rm pF}}
\newcommand{\mH}{{\rm mH}}
\newcommand{\muH}{{\rm \mu H}}
\newcommand{\uC}{\mu{\rm C}}
\newcommand{\nC}{{\rm nC}}
\newcommand{\MJ}{{\rm MJ}}
\newcommand{\kJ}{{\rm kJ}}
\newcommand{\kg}{{\rm kg}}
\newcommand{\g}{{\rm g}}
\newcommand{\m}{{\rm m}}
\newcommand{\s}{{\rm s}}
\newcommand{\ms}{\rm ms}
\newcommand{\us}{\mu{\rm s}}
\newcommand{\ns}{{\rm ns}}
\newcommand{\cm}{{\rm cm}}
\newcommand{\um}{\mu{\rm m}}
\newcommand{\mm}{{\rm mm}}
\newcommand{\nm}{{\rm nm}}
\newcommand{\km}{{\rm km}}
\newcommand{\G}{{6.67\times10^{-11}~\frac{\N~\m^2}{\kg^2}}}
\newcommand{\mpsps}{\frac{\m}{\s^2}}
\newcommand{\kgmsq}{\kg\cdot\m^2}
\newcommand{\Nm}{\N~\m}      % {\N\cdot\m}
\newcommand{\ihat}{{\hat{i}}}
\newcommand{\jhat}{{\hat{j}}}
\newcommand{\khat}{{\hat{k}}}
\newcommand{\Hz}{{\rm Hz}}
\newcommand{\kHz}{{\rm kHz}}
\newcommand{\MHz}{{\rm MHz}}
\newcommand{\kN}{{\rm kN}}
\newcommand{\red}{\color{red}}
%%%%%%%%%%%%%%%%%%%%%%%%%%

\newcommand{\lst}{\lstinline}
\newcommand{\bs}{\bigskip}
\newcommand{\Vo}{{V}_{out}}
\newcommand{\Vi}{{V}_{in}}
\newcommand{\fdb}{{f}_{3dB}}




%%%%%%%%%%%%%%%%%%%%%%%%%
% Some macros

\newcommand\myNIA[4]{%1: name of this amplifier, %2 start coordinate, %3 R1, %4 R2
\draw 
#2 coordinate(#1-in) to[short] ++(1,0)
node[op amp, noinv input up, anchor=+](#1-OA){\texttt{#1}}
(#1-OA.-) -- ++(0,-1) coordinate(#1-FB)
to[R=#3] ++(0,-2) node[ground]{}
(#1-FB) to[R=#4, *-] (#1-FB -| #1-OA.out) -- (#1-OA.out)
to [short, *-] ++(1,0) coordinate(#1-out)
;
}

\newcommand\myF[2]{
%1: name of this follower, %2 start coordinate
\draw
#2 to ++(2,0.5) coordinate(start)
(start) node[op amp](#1-OA){\texttt{F1}} (opamp) {}

(opamp.+) node[left] {}
to [short, *-] ++(-1,0) coordinate(#1-in)

(opamp.-) node[left] {}
to[short] ++(0,1)

(opamp.out) node[right] {}
to[short] ++(0,1.5)
to[short] ++(-2.4,0) 
(opamp.out) coordinate(#1-out)
;
}

\newcommand\myDividers[4]{
%1: exit of divider 1, %2 exit of divider 2
\draw
#3 coordinate(#4-in)
#1 to [battery, l=9 V] ++(0,2)
to [R, l=1 k$\Omega$] ++(2,0) coordinate(#4-out)
to [R, l=2 k$\Omega$] ++(0,-2) 

(2,2) to [short] #2

#1 to [short, -*] ++(9,0)

#3 to [R, l=1 k$\Omega$] ++(2,0)
to [R, l=2 k$\Omega$] ++(0,-2)
(8,2) to [short, -*] ++(1,0)
;
}

