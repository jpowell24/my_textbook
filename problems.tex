\section{Thesis Problems}

\textbf{Biology PhD Thesis}: Furthering our understanding of Ca$^{2+}$ spikes. (1) Discriminating neuronal Ca$^{2+}$ spikes in an EEG signal using Machine Learning. (2) Determining the contribution, if any, of glial Ca$^{2+}$ waves on an EEG or ECoG signals. One can begin by using GCaMP in tandem with ECoG recordings. 

\textbf{Applied Math PhD Thesis}: Computational modeling of a neuron's filtering ability. Begin with conduction-based models and work toward a reduced dynamical system (i.e., one which combines neuronal morphology in to a single variable). Seek out an LFP Power Law. Show that it maybe be applied to current systems. 

\textbf{Nobel Prize in Physics}: Solve the Navier-Stokes equations.


\textbf{Biology PhD Thesis}: An SBMA model including endogenous AR deletion in tissues with transgenic AR expression. Alternatively, a model with global endogenous AR knockout and removable transgenic AR expression. A stronger model resulting in neuron death (possible through more repeats) would be helpful as well. Test that excision from muscles with endogenous AR knockout and re-test phenotypes examined in Cortes et al. 2014. Similarly, after the model is made, test knockout in other tissues, including neurons and glia. The questions to address include the role of AR knockout (if any) in SBMA progression, and the relevance of other tissues to disease progression.

\endinput